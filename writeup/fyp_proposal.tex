\documentclass[11pt]{article}
\usepackage{amsmath,amssymb}
\usepackage{mathrsfs}
\usepackage{wasysym}
\usepackage{graphicx}
\usepackage{float}
\usepackage{pdflscape}
\usepackage{subcaption}
\usepackage{xcolor}
%%\floatstyle{boxed}
\usepackage{scalerel}
\usepackage[margin=3cm]{geometry}
\newcommand{\cell}[1]{\begin{minipage}[h]{3in} #1 \end{minipage}}
\newcommand{\tri}{\triangle}
\newcommand{\nonagon}{\,\vcenter{\hbox{\includegraphics[height=1.5ex]{figures/nonagon_symbol.png}}}\,}
\newcommand{\ngon}{\,\vcenter{\hbox{\includegraphics[height=1.5ex]{figures/ngon_symbol.png}}}\,}
\def\hex{\mathrel{%
    \mathchoice{\HEX}{\HEX}{\scriptsize\HEX}{\tiny\HEX}%
}}
\def\HEX{{%
    \setbox0\hbox{\varhexagon}%
    \rlap{\hbox to \wd0{\hss\raisebox{1 pt}{\scaleobj{0.6}{6}}\hss}}\box0
}}

\restylefloat{figure}
\begin{document}
%%\setcounter{secnumdepth}{-1}
\title{FYP Proposal: Example Choice \& Concept Learning in Group Theory}
\author{Andrew Lampinen \\ (Advisor: Dr.\ James McClelland)}
\date{}
\maketitle
\section{Introduction}
What is the purpose of a pedagogical example? As the word ``example'' suggests, examples are used to illustrate a broader concept, category, or idea. However, generally an example will not be perfectly representative, that is only some of its features will be category-general. In addition, students may have some preconceptions about the objects included in the example. Both of these factors may bias their inferences. Thus changing the superficial details of an example may alter what students learn about the category being represented. \\[11pt]
Furthermore, many concepts are built hierarchically on top of previously learned concepts. This idea has been considered for some time within cognitive psychology, e.g. \cite{Fischer1980}. Clearly the ability to learn higher-order concepts depends on understanding of the simpler concepts they are built upon. Thus we expect that the examples used to teach a concept can also affect student's understanding of the concepts which build upon it.\\[11pt]
In this project, I will explore these issues of how examples change learning of a concept, and the concepts built upon it. I plan to examine these issues within the area of mathematical cognition, specifically the learning of cyclic groups in group theory. There has been significant discussion of examples in mathematical pedagogy research. For example Mitchell Nathan's work has explored the effect ordering of abstract and applied problems in the curriculum has on learning mathematics \cite{Nathan2012}, Markant \& Gureckis have explored the effects of choosing or passively receiving examples when learning categories \cite{Markant2014}, and Larry Lesser has considered using counterintuitive examples to engage students \cite{Lesser1998}. More specifically, Kaminski et al. have investigated different presentations of cyclic groups \cite{Kaminski2008}, and these investigations initiated my interest in this topic. 

\subsection{Background: The Advantage(?) of Abstract Examples}
Kaminski and colleagues \cite{Kaminski2008} presented subjects with either a ``generic'' instantiation of a cyclic group of order 3, or a ``concrete'' one. (If you are unfamiliar with group theory, you may wish to review Appendix A at this point.) The examples are illustrated in figure \ref{kaminskitraining}. The generic presentation consists of some arbitrary geometric symbols, with enforced rules for combining them, and the concrete presentation consisted of an example with a narrative about combining cups of liquid, and finding the amount left over. There were two other concrete presentations (not shown) that were also used some training sessions (using pizza slices and tennis balls, respetively, as the concrete objects.) They trained subjects to perform the operation in either the generic presentation or one to three concrete presentations. They then showed subjects the transfer domain shown in figure \ref{kaminskitransfer}, where the objects of the group are replaced by toys in a children's game. The subjects were explicitly told that this followed the same rules as the earlier examples, and that they should try to use their knowledge to predict the correct answers. Kaminski found that the subjects who learned the generic presentation performed better at this transfer than the subjects who learned the concrete presentation(s). From this, they concluded that ``instantiating an abstract concept in a concrete, contextualized manner appears to constrain that knowledge and hinder the ability to recognize the same concept elsewhere'' \cite{Kaminski2008}. \\[11pt]
\begin{figure} \centering \begin{subfigure}{0.5\textwidth} \caption{Group Presentations} \label{kaminskitraining} \includegraphics[width=\textwidth]{figures/Kaminski2008Fig1.png} \end{subfigure} \\ \begin{subfigure}{0.5\textwidth} \caption{Transfer Domain} \label{kaminskitransfer} \includegraphics[width=\textwidth]{figures/KaminskiTransfer.png} \end{subfigure} \caption{Group Presentations from \cite{Kaminski2008}} \end{figure}
However, \cite{Kaminski2008} has been thoroughly criticized over these choices of the initial presentations and the transfer domain. For example, Matthew G. Jones pointed out that in the concrete presentation ``the feature in question ... is the physical objects that behave like quantities'' and the problems can be solved by adding and subtracting, whereas in the generic presentation ``the symbols used do not appear to represent \emph{quantities}, and are not combined,'' and the transfer task, similarly ``does not exhibit a quantitative feature; instead it is another version of the generic instatiation with a different contextualization.'' Thus he concludes that ``The transfer task is more similar to the generic instantiation than to the concrete ones'' \cite{Jones2009}. In a response to these critiques, Kaminski et al. asserted that the generic and transfer domains were not more similar, because after describing the domains to a set of subjects, without teaching them the rules for combinations, and asking them to rate the similarity between domains, and did not find any significant differences in similarity \cite{Kaminski2009}. However, not presenting the rules makes it difficult to claim this comparison truly captures the similarity between the domains. \\[11pt]
For example, one aspect of the domains which is different is the asymmetry between 1 and 2, based on the subjects previous arithmetic knowledge. Although in the abstract sense, it is clear that the generic domain and the concrete are isomorphic, in the generic domain the symmetry between the two non-identity elements is clear, circle circle = diamond, and diamond diamond = circle. While the rules that $1+1=2$ and $2+2=1$ do follow from the presentation in the numeric case, there is a fundamental asymmetry to the arithmetic interpretations of them (i.e. $1+1 = 2$ because $1/3$ cup two times makes $2/3$ cups, but $2+2 = 1$ because $2/3$ cup two times makes 1 and $1/3$ cups, and we throw away the full cup). We suspect this asymmetry may be to blame for the worse transfer performance, since students looking for a cue to map one object to 1 would not find any such specific cue. Similarly, if the notion of generators had been discussed in the study, students would probably have been biased to choose 1 as a generator, even though 2 is an equally good choice, whereas in the generic case there would be no such bias. Student's pre-conceptions about the examples in question can bias their understanding of the mathematical structure being presented. \\[11pt]
This idea that the learning is changed by the superficial appearance of the presentation is supported by De Bock et al., in their replication of Kaminski's study \cite{DeBock2011}. In this study, they compared the transfer from the generic domain to the concrete, and found that it was worse than the transfer from the concrete domain to a new concrete domain, or from an abstract to an abstract. Furthermore, they asked subjects to give a free response justifying their answer to a difficult problem, and graded it on the ideas that it contained. They found that generic-presentation group subjects were learning group-theoretic ideas better (although they still attained very little understanding of them), but that concrete-presentation group subjects were learning the ideas of modular arithmetic as well as some ideas of group theory. Thus, the choice of examples had an effect not just on transfer, but on the concepts being inferred. \\[11pt]
However, De Bock et al.\ did not thoroughly explore this concept. They asked only one question of subjects, and were only able to grade on the concepts the subjects explicitly mentioned, so they may have missed understanding which the subjects did not choose to explain, either because it seemed obvious or because they were not comfortable enough with the concept. Furthermore, they did not include some essential features which are present in real educational settings, most notably pedagogical explanation of the concepts in question. Finally, they only examined understanding within the context of a computational problem, instead of considering subjects learning at multiple levels of understanding. I believe the higher levels are also important, due to the hierarchical organization of group theoretic concepts. \\[11pt]
\subsection{Group Theory Concept Hierarchy}
Analogously with Van Hiele's levels of development in geometry \cite{Burger1986}, we propose some levels of understanding that occur while learning group theory:
\begin{description}
\item[Level 1 -- Individual Group Structure:] Students can perform group operation in a group (or groups).
\item[Level 2 -- Individual Groups Properties:] Students can recognize and explain group-theoretic elements in a group or groups, e.g. identities and inverses.
\item[Level 3 -- Families of Groups:] Students can understand the patterns shared by families of groups, e.g. cyclic or dihedral groups, and can reason about them without recourse to specific member groups.
\item[Level 4 -- Group Morphisms:] Students understand the concepts of homomorphisms, isomorphisms, and related concepts like normal subgroups, kernels, etc.
\item[Level 5 -- Category Theory:] Students can reason about groups in an abstract way, as a category with relations to other categories, without recourse to any specific groups or group families (except insofar as some families of groups form sub-categories of $\mathsf{Grp}$).
\end{description}
(Please note that these levels are speculation based on our own understanding of group theory, and we are not claiming it is the case that students will proceed linearly through these levels. Nevertheless, we think they provide a useful scaffold for the design of our experiment.) \\[11pt]
At each level of understanding, there are relationships back to the previous levels, and all understanding of higher levels is grounded upon understanding the lower ones. Indeed, Orit Hazzan has suggested that students learning a new concept in abstract algebra reduce the level of abstraction by relying on properties they understand in more concrete examples \cite{Hazzan1999}. In this experiment, we propose to manipulate the example used to illustrate level 1, and compare how subjects learn concepts at levels 1, 2, and 3.
 
\section{Proposed Methods} 
I propose to conduct an experiment investigating these effects, using two isomorphic representations of cyclic groups, which begin to address these issues by contrasting visual vs. non-visual representations, and differing levels of superficially apparent connections to basic arithmetic. The experiment would consist of a series of stages:  
\begin{enumerate}
\item Training on group operation
\item Training on concepts of identity and inverses
\item Training on concepts of generators
\item Test of ability to transfer concepts to a new cyclic group
\item Test of ability to formulate concepts at a general level about a family of groups
\end{enumerate}
We would train the subjects to perform a group operation using a cyclic group of order 6, and teach them the concepts of identities, inverses, and generators using this group. We would then test their transfer of these concepts to a cyclic group of order 9. (These groups were chosen in order to have enough elements for demonstrations of concepts like inverses, and to have sufficiently many generating and non-generating elements to make the generators question interesting). Finally, we would test subjects for understanding of the generic case by using a cyclic group with an unspecified order (i.e. order $n$). \\[11pt]
These tests address several levels of the hierarchy I proposed above. The learning of each group operation occurs at \textbf{Level 1}, the learning of identities, inverses and generators, as well as their transfer to a new explicit cyclic group occurs at \textbf{Level 2}, and the transfer to an generic cyclic group of order $n$ occurs at \textbf{Level 3}. (We will not investigate the higher levels within this study, although some of the concepts in \textbf{Level 4}, e.g. homomorphisms might not be too difficult to explain in the context of cyclic groups, and would provide an interesting ground for future investigations.)  
\subsection{Experimental Outline}
\subsubsection{Group Presentations}
Each experimental group would receive a different presentation of the cyclic groups. We have chosen these to compare an operational structure based on modular arithmetic, which is easily explained as a slight variation on regular addition, with a more visually concrete example, which allows subjects to develop a visual intuition, but which is not as directly familiar as standard arithmetic, although subjects may find analogies, e.g. to clocks. \\[11pt] 
For the modular arithmetic presentation, we would present the group operation as $+_n$, where $n$ is the order of the cyclic group, and we would explain to subjects that to compute $+_n$ you add the two numbers, and then subtract $n$ if your result is $n$ or bigger. For example, in our training presentation, we would use $+_6$, and show examples such as $4 +_6 4 = 8 - 6 = 2$. \\[11pt]
The visual presentation would be in the form of rotating an arrow around a polygon. We would write the group operation as $\hex$, an $n$ sided polygon containing the numeral $n$, and would provide the subjects with a visual aid like figure \ref{hexagonex}. (We would provide the subjects in this group with a manipulable arrow-in-polygon diagram on the screen for each problem.)
\begin{figure}[H] \centering \includegraphics[width=0.3\textwidth]{figures/hexagon_arrow.png} \caption{Order 6 polygon figure} \label{hexagonex} \end{figure} \noindent
We would explain to subjects that to compute $\hex$ you point the arrow in the hexagon to the first number, and then move it the second number of spaces clockwise. The number that the arrow points at is your result. For example, in our training presentation, we would use $\hex$, and show examples such as $4 \hex 4 = 2$, because 4 steps clockwise from 4 makes the arrow point at 2. \\[11pt]
After showing several examples, we would allow subjects to practice on problems until they got a set of 5 consecutive problems correct, at which point we would allow them to progress to the next stage of training. By tracking how many problems subjects must see to reach this point, we can estimate how difficult each representation is to learn.
\subsubsection{Indentities \& Inverses}
We would explain the concept of identity in precisely the same way in each presentation, by stating that 0 is the identity because when you combine it with anything, you get the same thing back. We would give two examples, e.g. $3 \hex 0 = 3$ and $0 +_6 5 = 5$. (Through the remainder of this paper, when presenting material that experimental groups will see in different forms, I will alternate notation use.) \\[11pt]
Similarly, we would explain the concept of inverses by saying something's inverse is what you need to combine with that thing to get back to the identity. For example, the inverse of 1 is 5, because $1 \hex 5 = 0$ and $5 +_6 1 = 0$. We would allow\ subjects to find inverses for all other group elements as pratice. 
\subsubsection{Generators}
Finally, we would teach the subjects the idea of generators, by explaining that a generator can make every other element of the group by combining with itself. For example, 1 is a generator under $+_6$, because $1 = 1$, $2 = 1+_6 1$, etc. However, 2 is not a generator under $+_6$, because $2 = 2$, $4 = 2 +_6 2$, $0 = 2+_6 2 +_6 2$, but there is no way to make 1, 3, or 5. Again, the only difference between the concept descriptions would be the way the numbers were written. We would then ask subjects whether each of the remaining elements generates the group.
\subsection{Transfer Test}
We would test the subjects transfer of concepts to the cyclic group of order 9, presented either as $+_9$, or $\nonagon$ with the visual aid in figure \ref{nonagonex}. 
\begin{figure}[H] \centering \includegraphics[width=0.3\textwidth]{figures/nonagon_arrow.png} \caption{Order 9 polygon figure} \label{nonagonex} \end{figure} \noindent
We would then ask the subjects questions to test their knowledge of the concepts outlined in each section above, namely:
\begin{itemize} 
\item A set of 10 problems with the group operation, e.g. $6 \nonagon 4 = ?$.
\item What is the identity under this operation? Why?
\item All 9 inverse problems for the group, with justification for one of them.
\item Which of 1, 2, 3, 5, and 6 generate the group, with justification for one generator and one non-generator.
\end{itemize}
\subsubsection{Generalization Test}
Finally, we would tell subjects we are now considering a order $n$ cyclic group, presented either as $+_n$, or $\ngon$ with the visual aid shown in figure \ref{ngonex}. (Unlike the other visual aids, in this one the arrow would rotate freely, and would not ``snap'' to the vertices, to avoid implicitly indicating a specific number of vertices to subjects.) 
\begin{figure}[H] \centering \includegraphics[width=0.4\textwidth]{figures/ngon_arrow.png} \caption{Order $n$ polygon figure} \label{ngonex} \end{figure} \noindent
We would then ask them the following questions: 
\begin{itemize}
\item What is the identity under $+_n$?
\item What is the inverse of an element $x$ under $\ngon$?
\item How can you tell if an element $x$ is a generator under $+_n$?
\item If an element $x$ is a generator under $\ngon$, is the inverse of $x$ a generator always, sometimes, or never?
\end{itemize}
\subsection{Hypothesis}
Our hypothesis is that there will be a difference in learning between the subject groups at several levels of understanding, and a presentation that is beneficial at one level may be deleterious at the next, and even on different types of questions within the same level. For instance, subjects learning from the numerical example may perform better at identifying generators, but worse at learning inverses. (We currently have no complete theory to predict which concepts will be more easily learned from which examples, but possibly the results of our experiment or a follow-up would allow us to speculate about this.)
\subsection{Implementation and Analysis details}
We would implement this experiment on Amazon's Mechanical Turk, using high-reputation subjects (over 85\% approval rate), using attention checks to ensure active participation, and using subject tracking (e.g. so we could run a follow-up or replication study on Mechanical Turk without having the same subjects participate and contaminate the results). The task would be developed using JSPsych framework with a custom plugin to integrate the interactive polygon diagrams for each question, and would be hosted externally using PsiTurk, and embedded in the Mechanical Turk page. We would perform standard statistical analyses for differences in scores between groups (i.e. $t$-tests) to test our hypotheses. 

\section{Conclusion}
We propose to explore the way pedagogical example choice in mathematics affects understanding of the concept being exemplified, and of concepts at higher levels of understanding, using elementary group theory as our test domain. We believe that this research will both yield interesting results, and provide the basis for future investigations into the more specific effects of example choice, and the acquisition of concept hierarchies.
 
\bibliography{fyp_citations}

\bibliographystyle{apalike}

\setcounter{secnumdepth}{-1}
\section{Appendix A: A Brief, Selective Introduction to Group Theory}
Groups are mathematical structures that provide us with a nice way of doing something like arithmetic with things besides the ordinary numbers, like symmetries of an object or permutations, or with smaller sets of ordinary numbers (as in this paper). They have applications throughout mathematics, physics, chemistry, and computer science. Here I present the formal definition of a group with informal intuitions in italics. A \textbf{group} consists of a set $G$ (\emph{some objects}) and a binary operation $*: G\times G \rightarrow G$ (\emph{a way of combining two objects to get another object}) such that:  
\begin{itemize}
\item $G$ is \textbf{closed} under $*$, that is $a*b \in G$ for all $a,b \in G$. (\emph{Combining two of the objects you started with gives you another of the objects you started with.}) 
\item $*$ is \textbf{associative}, $a*(b*c) = (a*b)*c$ for all $a,b,c \in G$. (\emph{It doesn't matter how you parenthesize the operation, just like addition or multiplication.})
\item There is an \textbf{identity} element $e \in G$ such that $\forall x \in G, e*x = x*e = x$. (\emph{There's something that when you combine it with anything else has no effect, just like multiplying by one gives you the same number back.})
\item Each element $x \in G$ has an \textbf{inverse} element $x^{-1} \in G$ such that $x*x^{-1} = x^{-1}*x = e$. (\emph{There's something you can combine with each element to get back to the identity, just like $2 \times 0.5 = 1$.})
\end{itemize}
For example, if we take $G$ to be the numbers less than $4$, $G = \{0,1,2,3\}$, and define a new operation $*$ by $$a*b = \begin{cases} a+b & \text{if } a+b < 4 \\ a+b-4 & \text{if } a+b \geq 4 \end{cases}$$
$G$ and $*$ form a group, called the \textbf{cyclic group of order 4}. For example, in this group $1*1 = 2$, $2 * 3 = 5-4 = 1$ because $5 \geq 4$, $3*1 = 4-4 = 0$, etc. $0$ is the identity in this group, because $0*x = x*0 = x$ for any of $0,1,2,3$. Furthermore, the inverse of $1$ in the group is $3$, because $1*3 = 4-4 = 0$, the inverse of $2$ is $2$, and so on.\\[11pt]
There is a great deal of structure to groups, far more than there is space to explain here. The only topic of interest for us beyond these simple properties will be the concept of \textbf{generators}. An element $x$ generates a group if every other element of the group can be written as $x*x*\cdots*x$ for some number of $x$s. For example, in our cyclic group of order 4, defined above, 1 is a generator of the group because $1 = 1, 2 = 1 * 1, 3 = 1 * 1 * 1, 4 = 1 * 1 * 1 * 1$. Similarly, 3 is a generator because $3 = 3, 2 = 3 * 3, 1 = 3 * 3 * 3, 0 = 3 * 3 * 3 * 3$. However, 2 is not a generator because $2 = 2, 0 = 2 * 2$, but there is no way to generate 1 or 3 using 2. This illustrates the only theorem we will give here: \\[11pt]
\textbf{Cyclic Group Generators Theorem:} In a cyclic group of order $n$, written as the integers $0$ to $n-1$, $x < n$ generates the group if and only if $x$ and $n$ are relatively prime (i.e. have no common factors except 1). \\[11pt] 
For more information on groups and group theory, see e.g. \cite{Lang2002}.
 
\end{document} 
