\documentclass[11pt]{article}
\usepackage{amsmath,amssymb}
\usepackage{mathrsfs}
\usepackage{wasysym}
\usepackage{graphicx}
\usepackage{float}
\usepackage{pdflscape}
\usepackage{subcaption}
\usepackage{xcolor}
%%\floatstyle{boxed}
\usepackage{scalerel}
\usepackage[margin=3cm]{geometry}
\newcommand{\cell}[1]{\begin{minipage}[h]{3in} #1 \end{minipage}}
\newcommand{\tri}{\triangle}
\newcommand{\nonagon}{\,\vcenter{\hbox{\includegraphics[height=1.5ex]{../figures/nonagon_symbol.png}}}\,}
\newcommand{\ngon}{\,\vcenter{\hbox{\includegraphics[height=1.5ex]{../figures/ngon_symbol.png}}}\,}
\def\hex{\mathrel{%
    \mathchoice{\HEX}{\HEX}{\scriptsize\HEX}{\tiny\HEX}%
}}
\def\HEX{{%
    \setbox0\hbox{\varhexagon}%
    \rlap{\hbox to \wd0{\hss\raisebox{1 pt}{\scaleobj{0.6}{6}}\hss}}\box0
}}

\restylefloat{figure}
\begin{document}
%%\setcounter{secnumdepth}{-1}
\title{FYP: Representations \& Concept Learning in Group Theory}
\author{Andrew Lampinen \\ (Advisor: Dr.\ James McClelland)}
\date{}
\maketitle
\section{Introduction}
What is the purpose of a pedagogical representation of a mathematical concept? As the word ``representation'' suggests, they are generally used to represent a broader concept, category, or idea. However, usually a representation will not be perfect, that is, only some of its features will be category-general. In addition, students may have some preconceptions about the objects included in the presentation. Both of these factors may bias the inferences students make about the concept being explained. Thus changing the way a concept is represented may alter what students learn. \\[11pt]
Furthermore, many concepts are built on top of previously learned concepts. (This idea has been considered for some time within cognitive psychology, e.g. \cite{Fischer1980}, and more specifically within math cognition, e.g. \cite{Hazzan1999}.) Clearly the ability to learn higher-order concepts depends on understanding of the simpler concepts they are built upon. Thus we expect that the representations used to teach a concept can also affect students' understanding of the concepts which build upon it.\\[11pt]
In this project, I explored these issues of how representations used to teach a concept change understanding of it, and the concepts built upon it. I examined these issues within the area of mathematical cognition, specifically the learning of cyclic groups in group theory. I have built upon the extensive past work on examples in mathematical pedagogy research. For example, Mitchell Nathan's work has explored the effect of ordering of abstract and applied problems in the curriculum has on learning mathematics \cite{Nathan2012}, Markant \& Gureckis have explored the effects of choosing or passively receiving examples when learning categories \cite{Markant2014}, and Larry Lesser has considered using counterintuitive examples to engage students \cite{Lesser1998}. More specifically, Kaminski et al. have investigated different presentations of cyclic groups \cite{Kaminski2008}, and these investigations provoked my interest in this topic. 

\subsection{Background: The Advantage(?) of Abstract Examples}
Kaminski and colleagues \cite{Kaminski2008} presented subjects with either a ``generic'' instantiation of a cyclic group of order 3, or a ``concrete'' one. (If you are unfamiliar with group theory, you may wish to review Appendix A at this point.) The examples are illustrated in figure \ref{kaminskitraining}. The generic representation consists of some arbitrary geometric symbols, with enforced rules for combining them, and the concrete representation consisted of an example with a narrative about combining cups of liquid, and finding the amount left over. There were two other concrete representations (not shown) that were also used in some experimental sessions (using pizza slices and tennis balls, respectively, as the concrete objects.) They trained subjects to perform the operation in either the generic representation or one to three concrete representations. They then showed subjects the transfer domain shown in figure \ref{kaminskitransfer}, where the objects of the group are replaced by toys in a children's game. The subjects were explicitly told that this followed the same rules as the earlier examples, and that they should try to use their knowledge to predict the correct answers. Kaminski found that the subjects who learned the generic representation performed better at this transfer than the subjects who learned the concrete representation(s). From this, they concluded that ``instantiating an abstract concept in a concrete, contextualized manner appears to constrain that knowledge and hinder the ability to recognize the same concept elsewhere'' \cite{Kaminski2008}. \\[11pt]
\begin{figure} \centering \begin{subfigure}{0.5\textwidth} \caption{Group Presentations} \label{kaminskitraining} \includegraphics[width=\textwidth]{../figures/Kaminski2008Fig1.png} \end{subfigure} \\ \begin{subfigure}{0.5\textwidth} \caption{Transfer Domain} \label{kaminskitransfer} \includegraphics[width=\textwidth]{../figures/KaminskiTransfer.png} \end{subfigure} \caption{Group Presentations from \cite{Kaminski2008}} \end{figure}\noindent
However, Kaminski and colleagues have been thoroughly criticized over these choices of the initial representations and the transfer domain. For example, Matthew G. Jones pointed out that in the concrete representations ``the feature in question ... is the physical objects that behave like quantities'' and the problems can be solved by adding and subtracting, whereas in the generic representation ``the symbols used do not appear to represent \emph{quantities}, and are not combined,'' and the transfer task, similarly ``does not exhibit a quantitative feature; instead it is another version of the generic instatiation with a different contextualization.'' Thus he concludes that ``The transfer task is more similar to the generic instantiation than to the concrete ones'' \cite{Jones2009}. In a response to these critiques, Kaminski et al. asserted that the generic and transfer domains were not more similar, because after describing the domains to a set of subjects, without teaching them the rules for combinations, and asking them to rate the similarity between domains, they did not find any significant differences in rated similarity \cite{Kaminski2009}. However, not presenting the rules makes it difficult to claim this comparison truly captures the similarity between the domains. \\[11pt]
For example, one aspect of the representations which is different is the asymmetry between 1 and 2, based on the subjects previous arithmetic knowledge. Although in the abstract sense, it is clear that the generic domain and the concrete are isomorphic, in the generic domain the symmetry between the two non-identity elements is clear, circle circle = diamond, and diamond diamond = circle. While the rules that $1+1=2$ and $2+2=1$ do follow from the presentation in the numeric case, there is a fundamental asymmetry to the arithmetic interpretations of them (i.e. $1+1 = 2$ because $1/3$ cup two times makes $2/3$ cups, but $2+2 = 1$ because $2/3$ cup two times makes 1 and $1/3$ cups, and we throw away the full cup to get back to $1/3$). We suspect this asymmetry may be to blame for the worse transfer performance, since students looking for a cue to map one object to 1 would not find any such specific cue. Similarly, if the notion of generators had been discussed in the study, students would probably have been biased to choose 1 as a generator, even though 2 is an equally good choice, whereas in the generic case there would be no such bias. Students' pre-conceptions about the examples in question can bias their understanding of the mathematical structure being presented. \\[11pt]
This idea that the learning is changed by the representation is supported by De Bock et al., in their replication of Kaminski's study \cite{DeBock2011}. In this study, they compared the transfer from the generic domain to the concrete, and found that it was worse than the transfer from the concrete domain to a new concrete domain, or from an abstract to an abstract. Thus, each representation was better for transferring to representations similar to it. Furthermore, they asked subjects to give a free response justifying their answer to a difficult problem, and graded it on the ideas that it contained. They found that generic-presentation group subjects were learning group-theoretic ideas better (although they still attained very little understanding of them), but that concrete-presentation group subjects were learning the ideas of modular arithmetic as well as some ideas of group theory. Thus, the choice of representation had an effect not just on transfer, but on the concepts being inferred. However, De Bock et al.\ did not thoroughly explore this idea. They asked only one question of subjects, and were only able to grade on the concepts the subjects explicitly mentioned, so they may have missed understanding which the subjects did not choose to explain, perhaps because it seemed obvious or because they were not comfortable enough with the concept.\\[11pt]
Furthermore, both Kaminski et al. and De Bock et al. omitted many essential features of real educational settings. They did not include much pedagogical explanation of the concepts in question, instead presenting the concepts as a set of rules that only had meaning by their relation to the subjects previous knowledge. They tested only on transfer to a mathematically isomorphic concept, whereas most examples in math instruction are intended to illustrate something more general (a teacher does not show students that $5+6 = 11$ just so they can add 5 and 6 in the future, but rather to illustrate the more general principles of addition, carrying, etc.) Because of this, they only examined understanding within the context of a computational problem, instead of considering subjects learning with multiple aspects of understanding, such as generalization, formalization, etc. (except for the question where De Bock et al. rated subjects understanding of more advanced concepts). We believe that these other aspects of understanding are important to consider when evaluating representations, because of the organization and relationships between mathematical concepts.
\subsection{Concept Organization \& Aspects of Understanding} %% TODO: Revise further?
In mathematics instruction in general, concepts are generally built upon the concepts that precede them. For example, multiplication can be explained as repeated addition, exponentiation as repeated multiplication, and then this process can be inductively generalized as Knuth's up-arrow notation in a further level of abstraction. Similarly, within group theory there are different aspects of reasoning which are generally built up sequentially. Instruction usually begins at the level of individual groups, with specific examples like the cyclic group of order 3 that Kaminski et. al used. It's then possible to think about families of groups that share certain properties, like cyclic or symmetric groups, and to consider more general relationships between groups, homo- and isomorphisms, quotients, etc. It's then possible to further abstract away from some specifics of group theory and reason more generally about categories, including the category of groups, $\mathsf{Grp}$. Each of these aspects of understanding is generally explained in terms of the previous ones.\\[11pt]
How does this organization of concepts affect learning? Orit Hazzan has suggested that students learning a new concept in abstract algebra reduce the level of abstraction by relying on properties they understand in more concrete examples \cite{Hazzan1999}, i.e. in the concepts they have previously learned. Hazzan addresses several ways of thinking about abstraction, as quality of relationships between the object of thought and the thinker, as a reflection of the process-object duality, and as a representation of the degree of complexity of the objects of thought (which is the closest. We think that each of these ideas has some bearing on the organization of concepts in abstract algebra, but we prefer to think of abstraction in terms of aspects of understanding which, while not independent, are not always tightly connected. Having a process for solving a type of problem does not necessarily imply the ability to formally explain the basis of the process. This can be true even if the process is generalizable to other similar problems. Indeed, it's our view that the process by which mathematical ideas are formalized, generalized, and abstracted is culturally constructed and depends upon our system of mathematics education to propagate it. \\[11pt]
Given this, it may be asked which representations are better for advancing which aspects of understanding. De Bock and colleagues evaluation of more advanced concepts hints at the idea that different representations may help with different aspects. Some representations may better prepare students to understand certain types of concepts, while others may help with other concepts, and still others may encourage the ability to formally express ideas. In this study, we attempted to move beyond earlier investigation of how representations affect understanding of a single process (the operation of a cyclic group of order 3), to investigate generalizations of this operation to a different group order, and concepts that can be built upon this operation (both as processes and formalisms). This will provide insight into the effects of how a concept is represented on other concepts related to it. 
%In service of understanding the organization of concepts in group theory, we propose some levels of understanding that students may attain, analogous to Van Hiele's levels of development in geometry \cite{Burger1986}:
%\begin{description}
%\item[Level 1 -- Individual Group Structure:] Students can perform the group operation in a group (or groups).
%\item[Level 2 -- Individual Groups Properties:] Students can recognize and explain group-theoretic elements in a group or groups, e.g. identities and inverses.
%\item[Level 3 -- Families of Groups:] Students can understand the patterns shared by families of groups, e.g. cyclic or dihedral groups, and can reason about them without recourse to specific member groups.
%\item[Level 4 -- Group Morphisms:] Students understand the concepts of homomorphisms, isomorphisms, and related concepts like normal subgroups, kernels, etc.
%\item[Level 5 -- Category Theory:] Students can reason about groups in an abstract way, as a category with relations to other categories, without recourse to any specific groups or group families (except insofar as some families of groups form sub-categories of $\mathsf{Grp}$).
%\end{description}
%(Please note that these levels are speculation based on our own understanding of group theory, and we are not claiming that students will necessarily proceed linearly through these levels. This is a topic which would justify its own investigation. Nevertheless, we think the idea of these levels provide a useful scaffold for understanding the design of our experiment.) \\[11pt]
%At each level of understanding, there are relationships back to the previous levels, and all understanding of higher levels is grounded upon understanding the lower ones. In this project, we manipulated the representation used to illustrate level 1, and compared how subjects learned concepts at levels 1, 2, and 3. 
\subsection{General Experimental Overview}
We conducted a series of experiments investigating the effects of representations, using two isomorphic representations of a cyclic group. One representation is based on a visuospatial manipulation involving counting around the verticies of a polygon, and the other is based on arithmetic. We used the group theoretic concepts of identities, inverses, and generators, as well as generalization from specific examples of cyclic groups to a generic cyclic group of order $n$ to investigate the effects of these representations on different aspects of understanding. 
\section{Experiment 1}
\subsection{Introduction}
In our first experiment, we explored whether the representations we chose produced differential understanding between the groups, and if so, for which aspects of understanding.
\subsection{Materials \& Methods} 
All materials can be found can be found on our github (https://github.com/lampinen/fyp cyclic groups), including complete versions of our experiments, which can be downloaded and run, or viewed using github's html preview. \\[11pt]
The experimental layout was as follows:
\begin{enumerate}
\item Training on group operation (order 6 group)
\item Training on concepts of identity, inverses, and generators
\item Test of ability to transfer concepts to a new cyclic group (order 9)
\item Test of ability to formulate concepts at a general level about a family of groups (order $n$)
\item Demographic and background questions
\end{enumerate}
We taught the subjects to perform a group operation using a cyclic group of order 6 (using one of two representations, between subjects), and then taught them the concepts of identities, inverses, and generators using this operation. The instruction for identities, inverses, and generators was the same between experimental groups (we did not need to refer to the specifics of the underlying operation). For example, for inverses we explained that ``the inverse of a number is the element that you combine with it to produce the identity.'' We then tested their transfer of these concepts to a cyclic group of order 9. (These group orders were chosen in order to have enough elements for demonstrations of concepts like inverses, and to have sufficiently many generating and non-generating elements to make the generator questions interesting). Finally, we tested subjects for understanding of the generic case by using a cyclic group with an unspecified order (i.e. order $n$). \\[11pt]
This design addresses several aspects of understanding. The learning of each group operation corresponds to a process-level understanding of a specific group. The concepts of identities, inverses and generators, are built upon this, and then the transfer to a cyclic group of a different order requires transfer of process-level understanding of how to find inverses, whereas the generic cyclic group of order $n$ requires the ability to formulate and express explicit, formal rules about the concepts learned. 
\subsubsection{Group Presentations}
Each experimental group received a different presentation of the cyclic groups. We have chosen these to compare a representation based on modular arithmetic, which is easily explained as a slight variation on regular arithmetic, with a more visually concrete representation based on counting around a polygon, which allows subjects to develop a visual intuition, but which is not as directly familiar as standard arithmetic, although subjects may find analogies, e.g. to clocks. \\[11pt] 
For the modular arithmetic presentation, we presented the group operation as $+_6$, and we explained to subjects that to compute $+_6$ you add the two numbers, and then subtract $6$ if your result is $6$ or bigger.\\[11pt]
For the polygon representation, we presented the group operation in the form of rotating an arrow around a polygon. We wrote the group operation as $\hex$, an $n$ sided polygon containing the numeral $n$, and provided the subjects with a visual aid like figure \ref{hexagonex}. The diagram that subjects were provided was interactive, so that they could click or click and drag to move the arrow around the polygon. (The diagram was provided on each problem in the experiment.)
\begin{figure}[H] \centering \includegraphics[width=0.3\textwidth]{../figures/hexagon_arrow.png} \caption{Order 6 polygon figure} \label{hexagonex} \end{figure} \noindent
We explained to subjects that to compute $\hex$ you point the arrow in the hexagon to the first number, and then move it the second number of spaces clockwise. The number that the arrow points at is your result. In our training presentation, we used $\hex$, and gave examples such as $4 \hex 4 = 2$, because 4 steps clockwise from 4 makes the arrow point at 2. \\[11pt]
After showing several examples, we allowed subjects to practice on 10 problems, and if their accuracy was below 80\%, they were given an additional 10 practice problems. On all of these problems, the subjects received feedback on their answers and an explanation of the correct answer. 
\subsubsection{Indentities \& Inverses}
Next, we explained the concept of identity by stating that 0 is the identity because when you combine it with anything, you get the same thing back. We gave two examples to illustrate this. (This, and all subsequent concepts, was explained in exactly the same way to the different experimental groups, except for the differences in the operation symbols used. For the remainder of the paper, when presenting material that both experimental groups saw, I will use either of the operation symbols.)\\[11pt]
Similarly, we explained the concept of inverses by saying something's inverse is what you need to combine with that thing to get to the identity. For example, the inverse of 1 is 5, because $1 \hex 5 = 0$ and $5 \hex 1 = 0$. We then allowed subjects to find inverses for all other group elements as practice, and subjects received feedback on their answers and an explanation of the correct answer.
\subsubsection{Generators}
Finally, we taught the subjects the idea of generators, by explaining that a generator can make every other element of the group by combining with itself. For example, 1 is a generator under $+_6$, because $1 = 1$, $2 = +_6 1$, etc. However, 2 is not a generator under $+_6$, because $2 = 2$, $4 = 2 +_6 2$, $0 = 2 +_6 2 +_6 2$, but there is no way to make 1, 3, or 5. We then asked subjects to find whether each of the remaining elements generates the group as practice, and provided them with feedback on their answers and an explanation of the correct answer.
\subsection{Transfer Test}
We first tested the subjects transfer of concepts to the cyclic group of order 9, presented to the modular group as $+_9$, or to the polygon group as $\nonagon$ with the visual aid in figure \ref{nonagonex}. 
\begin{figure}[H] \centering \includegraphics[width=0.3\textwidth]{../figures/nonagon_arrow.png} \caption{Order 9 polygon figure} \label{nonagonex} \end{figure} \noindent
We allowed the subjects one practice problem (with feedback) on the new operation, to ensure that they understood it. We then asked the subjects questions to test their knowledge of the concepts outlined in each section above, namely:
\begin{itemize} 
\item A set of seven problems with the group operation, e.g. $6 \nonagon 4 = ?$, with subjects asked to provide an explanation of their answers for two of them.
\item One problem on the identity under the operation, with explanation.
\item Three inverse problems for the group, with explanation for one of them.
\item Four generators questions (two generators, two non-generators), with explanation for one generator and one non-generator.
\end{itemize}
\subsubsection{Generalization Test}
Finally, we told subjects we were now considering a order $n$ cyclic group, presented to the modular group as $+_n$, and to the polygon group as $\ngon$ with the visual aid shown in figure \ref{ngonex}. (Unlike the other visual aids, in this one the arrow would rotate freely, and would not ``snap'' to the vertices, to avoid implicitly indicating a specific number of vertices to subjects.) 
\begin{figure}[H] \centering \includegraphics[width=0.4\textwidth]{../figures/ngon_arrow.png} \caption{Order $n$ polygon figure} \label{ngonex} \end{figure} \noindent
We then asked them the following questions: 
\begin{itemize}
\item What is the identity under $+_n$?
\item Two questions on giving formulas for inverses under $\ngon$, for 1 and for an arbitrary element $x$.
\item Two free-response questions on which elements are generators. 
\item Four true/false questions on which elements are generators, successively narrowing in on a correct statement about non-generators (If an element $x$ is not a generator under $+_n$, $x$ must be a multiple of a divisor of $n$.)
\item Three always/sometimes/never questions about generators. (E.g. If an element $x$ is a generator under $\ngon$, is its inverse a generator always, sometimes, or, never?) 
\end{itemize}
\subsection{Hypothesis}
Our hypothesis was that there would be a difference in learning between the subject groups in several of the aspects of undrstanding, and a presentation that is beneficial for one concept or aspect may be deleterious for another. (We had no a priori theory to predict which concepts would be more easily learned from which examples, so our results must be interpreted with this in mind.)
\subsubsection{Analysis}
We chose to analyse the data via a mixed-effects linear regression on the question-by-question scores of the subjects, with the fixed effects being question type, including the group order (6, 9, or n) in which it was presented; representation, polygon or modular; the interaction of those two; the effect of having a high math background, defined as algebra II, trigonometry, statistics, or above; and a random effect of subject. We excluded subjects who reported in the background section that they had used modular arithmetic or mathematical groups before. The results presented are taken from this analysis. (We did not compute multiple comparisons correction in our analyses, we instead validated them in later experiments. These results must be interpreted with this in mind.) 
\subsubsection{Implementation details}
We performed this experiment on Amazon's Mechanical Turk, using high-reputation subjects (over 85\% approval rate), and using subject tracking (so we could run follow-up and replication studys on Mechanical Turk without having the same subjects participate and contaminate the results). The task was developed using JSPsych framework with a custom plugin to integrate the interactive polygon diagrams for each question, hosted on Stanford's servers, and embedded in the Mechanical Turk page.  
\subsection{Results} %%TODO: Include full results or at least a better survey of them
\subsubsection{Operation}
There was no significant difference in the performance on the basic operation questions between the experimental groups (see Figure \ref{ex1_op}) in either the order 6 group ($\beta = 0.006$, $t = 0.18$, $p = 0.86$) or the order 9 group ($\beta = -0.009$, $t = -0.26$, $p = 0.79$). These are the questions where the wording varied in accordance with the different representations taught to the experimental groups. Both groups performed quite well, with over 90\% accuracy. 
\begin{figure}[H]
\centering
\begin{subfigure}[c]{0.4\textwidth}
\centering
\includegraphics[width=\textwidth]{figures/1/op_6_r.png}
\end{subfigure}
~
\begin{subfigure}[c]{0.4\textwidth}
\centering
\includegraphics[width=\textwidth]{figures/1/op_9_r.png}
\end{subfigure}
\caption{Experiment 1 -- Operation Results}
\label{ex1_op}
\end{figure} 
\subsubsection{Secondary Concepts}
Despite the fact that the experimental groups did not significantly differ on performing the operation, and the instruction on the secondary concepts was identical between experimental groups, we did find significant differences on some of these secondary concepts. First, for inverses the polygon group was significantly worse at finding the inverse of non-zero elements in the order 9 group ($\beta = -0.169$, $t=-3.73$, $p < 0.001$), but better at finding the inverse of zero (see Figure \ref{ex1_in}) in the order 6 group ($\beta = 0.66$, $t=9.13$, $p < 0.001$). (Unfortunately we did not include an inverse of zero question in the order 9 group in this experiment, so we only have data from the order 6 group.)
\begin{figure}[H]
\centering
\begin{subfigure}[c]{0.4\textwidth}
\centering
\includegraphics[width=\textwidth]{figures/1/in_NZ_r.png}
\end{subfigure}
~
\begin{subfigure}[c]{0.4\textwidth}
\centering
\includegraphics[width=\textwidth]{figures/1/in_Z_r.png}
\end{subfigure}
\caption{Experiment 1 -- Inverse Results}
\label{ex1_in}
\end{figure}\noindent 
Second, the polygon group was significantly better at identifying elements that were generators (order 6: $\beta = 0.164$, $t = 2.27$, $p = 0.02$; order 9: $\beta = 0.184$, $t = 3.45$, $p < 0.001$). However, the modular group performed better at the T/F questions about generators in the order $n$ group ($\beta = -0.11$, $t = -2.60$, $p = 0.009$). (See Figure \ref{ex1_gen}.)
\begin{figure}[H]
\centering
\begin{subfigure}[c]{0.4\textwidth}
\centering
\includegraphics[width=\textwidth]{figures/1/gen_T_r.png}
\end{subfigure}
~
\begin{subfigure}[c]{0.4\textwidth}
\centering
\includegraphics[width=\textwidth]{figures/1/gen_TF_r.png}
\end{subfigure}
\caption{Experiment 1 -- Generator Results (we have elected not to draw a chance line on this graph, since we believe it would be misleading as subjects generally performed above chance on some questions, and below chance on others)}
\label{ex1_gen}
\end{figure}\noindent 
Although there was a significant difference between the representations on the True/False questions, performance was not too far above chance, and generally performance was at floor in the order $n$ cyclic group. For example, the results from the Always/Sometimes/Never questions are shown in Figure \ref{ex1_ASN}.
\begin{figure}[H]
\centering
\includegraphics[width=0.4\textwidth]{figures/1/gen_ASN_r.png}
\caption{Experiment 1 -- Always/Sometimes/Never question results}
\label{ex1_ASN}
\end{figure}\noindent
This poor performance suggests that subjects were mostly unable to formalize their understanding in a generic way, despite being able to generalize it quite well to a group of different order. 
\subsection{Discussion}
The results show that, despite the fact that the experimental groups did not differ at learning the initial operation, they did differ in their ability to understand the subsequent concepts built upon it. Furthermore, one representation was not generally ``better'' than the other, they both had strengths and weaknesses, and neither seemed to encourage formalization particularly well. Thus our hypotheses were confirmed, at least in this initial experiment. However, because we did not have a priori hypotheses about which representation would be better for which questions, these results are not conclusive. \\[11pt]
Although we have not fully explored these factors, we do have some post-hoc hypotheses for the pattern of results we observed, based on responses to problems where we asked the subjects to explain their answers. We have not been able to quantitatively assess these hypotheses in detail (although we have some slightly relevant analysis in experiment 2), so they should be treated as somewhat speculative at this point.\\[11pt]
\textbf{Inverses:} It appears that the modular representation may have cued the subjects to recognize an algorithm for finding most of the inverses, namely that the inverse of $x$ can be found by subtracting $x$ from the group order. For example, under $+_6$, the inverse of $2$ is $4$, and $6-2 = 4$. This would not be as obvious for the polygon group, which mostly found inverses by counting, which is less reliable. However, the algorithm breaks down when $x = 0$, because the inverse of $0$ is $0$, not the group order. We suspect that subjects in the polygon group made more mistakes counting than the modular group on inverses of non-zero elements, but the modular group subjects were misled by this algorithm when computing the inverse of zero. \\[11pt]
\textbf{Generators:} There is a spatial structure to the generator questions in the polygon case which may assist in solving them, for example finding if $5$ is a generator on the nonagon, we get the sequence $5 \rightarrow 1 \rightarrow 4 \rightarrow 2 \rightarrow \cdots$. It might be more clear to someone seeing the polygon how precisely this sequence would fill in the gaps to generate all the numbers. By contrast, in the order $n$ case, it is difficult or impossible to use these spatial intuitions, so the more formulaic understanding cued by the modular representation might be beneficial.

\section{Experiment 2}
\subsection{Introduction}
In our second experiment, we had two goals. First, we wanted to replicate the results of our first experiment with a planned analysis (to ensure that the effects were real and not just chance variation, since we didn't have \textit{a priori} hypotheses about which representation would be superior for which types of questions). Second, we wanted to explore whether we could teach both representations to the subjects and encourage them to integrate the polygon and modular methods into a hybrid representation (while keeping total instruction time approximately the same), and if this would cause improved performance across all types of questions. %%TODO: Motivate in greater detail and explain 
\subsection{Materials \& Methods} 
All materials can be found can be found on our github (https://github.com/lampinen/fyp cyclic groups), including complete versions of our experiments, which can be downloaded and run, or viewed using github's html preview. The materials and methods were identical to experiment 1, except:\\[11pt]
\textbf{Hybrid group:} We added a hybrid group, which was taught both representations initially (with polygon labeled as ``polygon'', and modular labeled as ``arithmetic''). We encouraged subjects to integrate these representations by adding a question and explanation about how the different representations worked the same, giving them a chance to reflect on the isomorphism between them (though we did not use that word). Furthermore, we asked subjects on alternating operation practice questions to use the different representations, to encourage them to achieve facility with each representation. (The hybrid group saw the same number of practice questions as the other experimental groups.)\\[11pt]
\textbf{Added questions:} Because we added a question to the hybrid group asking them how the representations worked the same, we added questions to the other experimental groups asking and explaining how the operations worked, giving them a chance to reflect on the representation. We also added a question about the inverse of zero in the order 9 group, to see if subjects in the modular condition performed better on it after getting feedback on their previous answer. \\[11pt]
\textbf{Other:} Typo fixes, wording adjustments, and other small changes to the experiment. As mentioned above, the versions can be compared on our GitHub.\\[11pt]
(All implementation details were as in experiment 1.)
\subsection{Hypotheses}
We hypothesized that we would replicate the results of our first experiment, namely: 
\begin{itemize} 
\item The modular and polygon groups would not differ significantly in their learning of the operation.
\item The modular group would be significantly better than the polygon group at finding the inverse of non-zero elements.
\item The polygon group would be significantly better than the modular group at finding the inverse of zero.
\item The polygon group would be significantly better than the modular group at identifying elements that are generators in the specific groups.
\item The modular group would be significantly better than the polygon group at answering T/F questions about generators in the order $n$ group.
\end{itemize}
Furthermore, we hypothesized that the hybrid group would achieve approximately the maximum performance of the two groups, i.e.:
\begin{itemize}
\item The hybrid group would be significantly better than the modular group whenever the polygon group performed better.
\item The hybrid group would not significantly differ from the modular group when the modular group performed better.
\end{itemize}
This can be contrasted with other possible predictions. One possibility is that seeing both representations would simply confuse or overload the subjects, and they would perform worse on every type of question, resulting in them being significantly worse at every question type. Another possibility is that subjects would just pick one representation and use it exclusively, and perform as though they were subjects in that representation group. This, and possibilities such as subjects randomly picking a representation to use on each question, would result in patterns of data where the hybrid group appeared to perform at the average of the other two groups. (Of course, there may be individual differences, and some subjects may achieve maximal performance while others are simply confused, which could also produce a similar effect.)
\subsubsection{Analysis}
We analyzed the data via a planned mixed-effects linear regression on the question-by-question scores of the subjects, with the fixed effects being question type, including the group order (6, 9, or n) in which it was presented; representation, (polygon, modular, or hybrid); the interaction of those two; the effect of having a high math background (as in experiment 1); and a random effect of subject. We excluded subjects who reported in the background section that they had used modular arithmetic or mathematical groups before. The results presented are taken from this analysis.
\subsection{Results}
\subsubsection{Operation}
We replicated our result that there was no significant difference in the performance on the basic operation questions between the modular and polygon experimental groups (see Figure \ref{ex2_op}) in either group order (order 6: $\beta = 0.02$, $t = 0.735$, $p = 0.46$; order 9: $\beta = 0.02$, $t = 0.63$, $p = 0.53$). Furthermore, there was no significant difference between the hybrid and modular groups (order 6: $\beta = -0.02$, $t = -0.72$, $p = 0.47$; order 9: $\beta = 0.04$, $t = 1.10$, $p = 0.27$).
\begin{figure}[H]
\centering
\begin{subfigure}[c]{0.4\textwidth}
\centering
\includegraphics[width=\textwidth]{figures/2/op_6_r.png}
\end{subfigure}
~
\begin{subfigure}[c]{0.4\textwidth}
\centering
\includegraphics[width=\textwidth]{figures/2/op_9_r.png}
\end{subfigure}
\caption{Experiment 2 -- Operation Results}
\label{ex2_op}
\end{figure} 
\subsubsection{Secondary Concepts}
We replicated our results that the modular group performed better at inverses of non-zero elements (order 6: $\beta = -0.11$, $t = -2.59$, $p = 0.009$; order 9: $\beta = -0.07$, $t = -1.48$, $p = 0.13$), and the polygon group performed better at inverse of zero questions (order 6: $\beta = 0.68$, $t = 8.842$, $p < 0.001$; order 9: $\beta = 0.44$, $t = 5.80$, $p < 0.001$). (Note that the modular group performed worse than the polygon group even in the order 9 group, once they had already seen an example with feedback.) The hybrid group was not significantly worse than the modular group at inverses of non-zero elements (order 6: $\beta = -0.007$, $t = -0.173$, $p = 0.86$; order 9: $\beta = 0.08$, $t = 1.56$, $p = 0.12$), and was significantly better at inverses of zero (order 6: $\beta = 0.46$, $t = 5.95$, $p < 0.001$; order 9: $\beta = 0.38$, $t = 4.91$, $p < 0.001$). (See Figure \ref{ex2_in}.) However, it is clear that the hybrid group is not initially performing as well as the polygon group on this question, so there is still room for improvement. \\[11pt] 

\begin{figure}[h]
\centering
\begin{subfigure}[c]{0.4\textwidth}
\centering
\includegraphics[width=\textwidth]{figures/2/in_NZ_r.png}
\end{subfigure}
~
\begin{subfigure}[c]{0.4\textwidth}
\centering
\includegraphics[width=\textwidth]{figures/2/in_Z_r.png}
\end{subfigure}
\caption{Experiment 2 -- Inverse Results}
\label{ex2_in}
\end{figure} 
Furthermore, we replicated our result that the polygon group is better at identifying generators in the order 9 group ($\beta = 0.19$, $t = 3.37$, $p < 0.001$), and we replicated our result that the modular group is better at T/F questions about generators ($\beta = -0.12$, $t = -2.71$, $p = 0.007$). The hybrid group is significantly better than the modular group at identifying generators in the group order where our previous finding replicated ($\beta = 0.25$, $t = 4.31$, $p < 0.001$), and is not significantly worse at the T/F questions ($\beta = -0.042$, $t = -0.98$, $p = 0.33$). (See Figure \ref{ex2_gen}.) However, it appears that the hybrid group may not be achieving performance completely on par with the modular group on these questions.
\begin{figure}[H]
\centering
\begin{subfigure}[c]{0.4\textwidth}
\centering
\includegraphics[width=\textwidth]{figures/2/gen_T_r.png}
\end{subfigure}
~
\begin{subfigure}[c]{0.4\textwidth}
\centering
\includegraphics[width=\textwidth]{figures/2/gen_TF_r.png}
\end{subfigure}
\caption{Experiment 2 -- Generator Results}
\label{ex2_gen}
\end{figure}\noindent 
\subsection{Post-hoc analyses}
We conducted a number of post-hoc analyses in an attempt to understand the differences between the groups, and the pattern of performance of the hybrid group. 
\subsubsection{Diagram use}
We hypothesized above that the polygon group's superior performance on identifying generators was due to the ability to use the spatial structure of the polygon to more easily visualize the elements generated by an element. One possible prediction of this hypothesis would be that within the polygon group, interaction with the diagram might be predictive of success on these questions. (Of course, we could only record the interactions with the mouse, while many subjects may have just gazed or pointed at the diagram to use it in their thinking. Furthermore, the use of the diagram may be confounded with overall engagement. Our results must be interpreted with this in mind.) \\[11pt]
We performed a mixed-model logistic regression on data from the polygon and hybrid subjects, predicting correct answers by whether or not they used the diagram (and a random effect of subject). We found that using the diagram was significantly predictive of success on the questions ($\beta=1.90$, $z = 2.76$, $p = 0.006$). Furthermore, this effect was present even when controlling for reaction time ($\beta = 1.62$, $z = 2.26$, $p = 0.024$), which might suggest that engagement alone wasn't the driving factor, and the effect was significant or trending within the polygon and hybrid conditions individually, suggesting that both benefitted. \\[11pt]
Using analogous mixed-model logistic regressions across the full data from the hybrid and polygon groups, we found that on all questions in the experiment (not just generator questions) that using the diagram was significantly predictive of success ($\beta = 1.26$, $z = 6.93$, $p < 0.001$), even when controlling for reaction time ($\beta = 1.42$, $z = 7.53$, $p < 0.001$). However, the estimated effect sizes were slightly smaller than for the generator questions. Further data is needed to clarify this pattern.
\subsubsection{Hierarchical modeling of hybrid subjects}
One of our main goals for this experiment was to test the idea of a hybrid group which would attain better performance than either the polygon or modular group individually, by integrating the material. Although we found the hybrid group did perform better, it did not seem to achieve best-of-both-worlds performance. One explanation for this might be that some subjects were just picking one representation and using it consistently, while others were really integrating both and performing optimally (at the maximal level of the two). We attempted to model this with a hierarchical model by assuming that the data were generated by the following process: 
\begin{enumerate}
\item With probability $\theta$, the subject would integrate the material, and would perform optimally in the sense that their data would be best fit by assuming they picked the optimal representation on each question (or equivalently, that their regression coefficients were the element-wise maximum of the regression coefficients of the two other groups). 
\item If the subjects did not integrate (probability $1-\theta$), they would pick the polygon representation with probability $\phi$, and the modular representation with probability $1-\phi$, and their data would be best fit by the coefficients for the respective group.
\end{enumerate}
We fit this model to the data using the expectation-maximization algorithm, and estimated that $\theta = 0.53, \phi = 0.38$ (log-likelihood $LL=-821.9$), so the data are best fit under this model by assuming that a little more than half the subjects are perfectly integrating, and those that aren't are choosing the modular representation over the polygon almost two-thirds of the time. We used the Aikake Information Criterion (AIC) to compare this model ($AIC = 1647.8$) to models where all subjects chose modular ($AIC=1829.6$), all chose polygon ($AIC = 1720.5$), where no subjects integrated i.e. a fixed $\theta = 0$ and fit $\phi = 0.65$ ($AIC = 1678.2$), and a model where all subjects integrated i.e. $\theta = 1.0$ ($AIC = 1689.9$). The full model is significantly better than any of these comparison models. However, there are many other possible ways people could use the two representations (such as picking arbitrarily on each question), so further investigation is needed.
\subsubsection{Analysis of hybrid subject word use}
We also attempted to investigate the hybrid subjects' use of different representations by the presence of words in their explanations that corresponded to one of the representations or another. First, we hand-crafted classifiers for the polygon and modular condition, respectively, based on the most common words that were specific to the condition (Polygon condition: move, spaces, clockwise, steps, arrow, hexagon, nonagon, ngon, spots, positions; Modular condition: add, subtract, plus, minus, +). We counted the occurrences of each of these words in each explanation, and then classified it as polygon if the polygon-word-count was higher, and as modular if the modular-word-count was higher. This produced highly specific, though not especially sensitive, classifiers for the conditions, see table \ref{wordusevaltable} for comparisons of classifier estimate of condition versus true condition across all questions with explanations in experiment 1. 
\begin{table}[h] 
\centering
    \begin{subtable}[c]{\textwidth}
	\centering
	\begin{tabular}{|r|c c|}
	    \hline   & not classified polygon & classified polygon \\ 
	    \hline actually modular & 499 & 1 \\
	    actually polygon & 341 & 159 \\ \hline 
	\end{tabular}
	\caption{Polygon classifier}
    \end{subtable}
    \newline \vspace{1em}\newline 
    \begin{subtable}[c]{\textwidth}
	\centering
	\begin{tabular}{|r|c c|}
	    \hline   & not classified modular & classified modular \\ 
	    \hline actually modular & 380 & 120 \\
	    actually polygon & 496 & 4 \\ \hline 
	\end{tabular}
	\caption{Modular classifier}
    \end{subtable}
\caption{Validation of hand-crafted word-use classifiers}
\label{wordusevaltable}
\end{table} 
Using these classifiers on the hybrid data from experiment 2, we found that the hybrid subjects were classified as using more polygon words on 58 questions total across all subjects, and more modular words on 33 questions. We compared scores across classifier groups, and found some intriguing patterns, see table \ref{worduseresultstable}. On questions about the inverse of non-zero elements, hybrid subjects rated as using more polygon words appeared more accurate than those not using more polygon words, and those using more modular words appeared less accurate than those not using more modular words (table \ref{polyclassintable} \& \ref{modclassintable}). Similarly, on questions where they had to identify a generator, hybrid subjects rated as using more polygon words appeared more accurate than those not using more polygon words, and those using more modular words also appeared more accurate than those not using more modular words, but more subjects used polygon words than modular (table \ref{polyclassgentable} \& \ref{modclassgentable}). \\[11pt]
\begin{table}[h] 
\centering
    \begin{subtable}[c]{\textwidth}
	\centering
	\begin{tabular}{|r|c c|}
	    \hline   & score = 0 & score = 1 \\ 
	    \hline not classified polygon & 6 & 30 \\
	    classified polygon & 0 & 10 \\ \hline 
	\end{tabular}
	\caption{Inverse of non-zero questions: polygon classifier}
	\label{polyclassintable}
    \end{subtable}
    \newline \vspace{1em}\newline 
    \begin{subtable}[c]{\textwidth}
	\centering
	\begin{tabular}{|r|c c|}
	    \hline   & score = 0 & score = 1 \\ 
	    \hline not classified modular & 4 & 37 \\
	    classified modular & 2 & 3 \\ \hline 
	\end{tabular}
	\caption{Inverse of non-zero questions: modular classifier}
	\label{modclassintable}
    \end{subtable}
    \newline \vspace{1em}\newline 
    \begin{subtable}[c]{\textwidth}
	\centering
	\begin{tabular}{|r|c c|}
	    \hline   & score = 0 & score = 1 \\ 
	    \hline not classified polygon & 45 & 38 \\
	    classified polygon & 0 & 9 \\ \hline 
	\end{tabular}
	\caption{Generator questions: polygon classifier}
	\label{polyclassgentable}
    \end{subtable}
    \newline \vspace{1em}\newline 
    \begin{subtable}[c]{\textwidth}
	\centering
	\begin{tabular}{|r|c c|}
	    \hline   & score = 0 & score = 1 \\ 
	    \hline not classified modular & 45 & 45 \\
	    classified modular & 0 & 2 \\ \hline 
	\end{tabular}
	\caption{Generator questions: modular classifier}
	\label{modclassgentable}
    \end{subtable}
\caption{Word-use classifier results}
\label{worduseresultstable}
\end{table} 
These results should not be read too strongly, because they are based on a few data points, because we did not ask for explanations on every question, and many subjects did not use specific words in their explanations. In part, this may be due to our design, by attempting to homogenize the language we used when presenting more advanced concepts to subjects, and remove references to the underlying operation, we implicitly encouraged subjects to give explanations that used representation-agnostic language (e.g. ``5 can make all the other elements'' instead of ``using 5 we can reach each other point on the nonagon''). We also tried using more general classifiers that were derived directly from the data, using the 20 or 40 most selective words and estimating the posterior probabilities from these, but while these classifiers were more sensitive, they were significantly less specific, so we think the interpretation of those results are even more difficult. 
\subsection{Discussion}
Our results generally replicated those of the original experiment. The only results that failed to replicate were the polygon group performing better at identifying generators in the order 6 group, and the polygon group performing worse at the inverse of non-zero elements in the order 9 group. However, both results were trending in the correct direction, and since each did replicate in the other group order, it is possible that there is a real effect that these experiments were not sufficiently powered to detect reliably. A further experiment to verify this would be ideal. However, the general pattern of results remains, and thus our original conclusions that the representation can have effects on learning of more abstract concepts remain. \\[11pt]
Furthermore, the hybrid group approached best of both worlds performance, although it did not appear to fully reach that level, so there is still progress to be made. Our post-hoc analyses suggest that some subjects may have integrated the results better than others, so there may be some individual variation. However, these results at least begin to suggest that the use and integration of multiple representations could be an important direction to pursue in future math pedagogy research. \\[11pt]
Finally, in our post-hoc analyses we produced some corroborating evidence that the polygon condition is better at identifying generators because of the spatial structure of generation on the polygon, by showing that using the diagram improved performance on these questions. 


\section{Experiment 3}
\subsection{Introduction}
In our third experiment, we wanted to replicate the results of our previous experiment, as well as further exploring the factors that enhanced the hybrid group's performance. In order to explore this, we added questions for the hybrid group (after the main experiment had been completed), in which they described the extent to which they had used each representation on the previous question. 
\subsection{Materials \& Methods} 
All materials can be found can be found on our github (https://github.com/lampinen/fyp cyclic groups), including complete versions of our experiments, which can be downloaded and run, or viewed using github's html preview. The materials and methods were identical to experiment 2, except:\\[11pt]
\textbf{Which-representation questions:} We added four questions for the hybrid subjects in which they answered a question, and then subsequently indicated on likert scales for each representation the degree to which they had used it on the preceding question. After this, they were presented a text box and asked to describe in as much detail as possible how they had used each representation in solving the question. We added one question for each of the four question types where we previously observed an effect, inverse of zero, inverse of non-zero elements, identifying generators, and answering T/F questions about generators. \\[11pt]
\textbf{Other:} Typo fixes, wording adjustments, and other small changes to the experiment. As mentioned above, the versions can be compared on our GitHub.\\[11pt]
(All implementation details were as in experiments 1 \& 2.)
\subsection{Hypotheses}
We hypothesized that we would replicate the results of second experiment, namely: 
\begin{itemize} 
\item The modular and polygon groups would not differ significantly in their learning of the operation.
\item The modular group would be significantly better than the polygon group at finding the inverse of non-zero elements.
\item The polygon group would be significantly better than the modular group at finding the inverse of zero.
\item The polygon group would be significantly better than the modular group at identifying elements that are generators in the specific groups.
\item The modular group would be significantly better than the polygon group at answering T/F questions about generators in the order $n$ group.
\end{itemize}
Furthermore, we hypothesized that the hybrid group would achieve approximately the maximum performance of the two groups, i.e.:
\begin{itemize}
\item The hybrid group would be significantly better than the modular group whenever the polygon group performed better.
\item The hybrid group would not significantly differ from the modular group when the modular group performed better.
\end{itemize}
This can be contrasted with other possible predictions. One possibility is that seeing both representations would simply confuse or overload the subjects, and they would perform worse on every type of question, resulting in them being significantly worse at every question type. Another possibility is that subjects would just pick one representation and use it exclusively, and perform as though they were subjects in that representation group. This, and possibilities such as subjects randomly picking a representation to use on each question, would result in patterns of data where the hybrid group appeared to perform at the average of the other two groups. (Of course, there may be individual differences, and some subjects may achieve maximal performance while others are simply confused, which could also produce a similar effect.)\\[11pt]
Finally, for the questions where we had the hybrid subjects describe which representation they used, we hypothesized that where the polygon subjects performed better, using the polygon representation would be significantly predictive of success or using the modular representation would be significantly predictive of failure, and vice versa for the questions where the modular subjects performed better.
\subsubsection{Analysis}
We analyzed the data via a planned logistic regression on the question-by-question scores of the subjects, which we bootstrapped across 10,000 resamples of the subjects, with the predictors being question type, including the group order (6, 9, or n) in which it was presented; representation, (polygon, modular, or hybrid); the interaction of those two; the effect of having a high math background (as in experiment 1). We excluded subjects who reported in the background section that they had used modular arithmetic or mathematical groups before. We used the inclusion of zero in the 95\% confidence intervals for the predictors as our signficance testing. This analysis was pre-registered on the Open Science Framework (https://osf.io/5gthx/). The results presented are taken from this analysis. For the hypotheses about the which-representation questions, we used logistic regression predicting score on the question by the ratings of representation used.
\subsection{Results} %%TODO: Stats and fix plots
\subsubsection{Operation}
We replicated our result that there was no significant difference in the performance on the basic operation questions between the modular and polygon experimental groups (see Figure \ref{ex3_op}) in either group order (order 6: $\beta$ 95\%-CI = $[-2.55,0.15]$, $p > 0.05$; order 9: $\beta$ 95\%-CI = $[-1.59,0.11]$, $p > 0.05$), and our result that there was no significant difference between the hybrid and modular groups (order 6:  $\beta$ 95\%-CI = $[-2.13,0.13]$, $p > 0.05$; order 9: $\beta$ 95\%-CI = $[-1.90,0.17]$, $p > 0.05$).
\begin{figure}[H]
\centering
\begin{subfigure}[c]{0.4\textwidth}
\centering
\includegraphics[width=\textwidth]{figures/3/op_6_r.png}
\end{subfigure}
~
\begin{subfigure}[c]{0.4\textwidth}
\centering
\includegraphics[width=\textwidth]{figures/3/op_9_r.png}
\end{subfigure}
\caption{Experiment 3 -- Operation Results}
\label{ex3_op}
\end{figure} 
\subsubsection{Secondary Concepts}
We replicated our results that the modular group performed better than the polygon group at inverses of non-zero elements only in the order 9 group, although the effect was in the correct direction in the order 6 group (order 6: $\beta$ 95\%-CI = $[-1.14,0.24]$, $p > 0.05$; order 9: $\beta$ 95\%-CI = $[-1.74,-0.27]$, $p < 0.05$). We replicated our result that the polygon group performed better at inverse of zero questions (order 6: $\beta$ 95\%-CI = $[1.93,3.55]$, $p < 0.05$; order 9: $\beta$ 95\%-CI = $[1.72,3.60]$, $p < 0.05$). The hybrid group was not significantly worse than the modular group at inverses of non-zero elements in either group, although it was trending worse in the order 6 group (order 6: $\beta$ 95\%-CI = $[-1.92,0.01]$, $p > 0.05$; order 9: $\beta$ 95\%-CI = $[-1.96,0.24]$, $p > 0.05$). We replicated our result that the hybrid group was significantly better at inverses of zero only in the order 6 group, although the effect was trending in the order 9 group (order 6: $\beta$ 95\%-CI = $[0.21,2.65]$, $p < 0.05$; order 9: $\beta$ 95\%-CI = $[-0.01,2.09]$, $p > 0.05$). (See Figure \ref{ex3_in}.) \\[11pt]
\begin{figure}[ht]
\centering
\begin{subfigure}[c]{0.4\textwidth}
\centering
\includegraphics[width=\textwidth]{figures/3/in_NZ_r.png}
\end{subfigure}
~
\begin{subfigure}[c]{0.4\textwidth}
\centering
\includegraphics[width=\textwidth]{figures/3/in_Z_r.png}
\end{subfigure}
\caption{Experiment 3 -- Inverse Results}
\label{ex3_in}
\end{figure} 
Furthermore, we replicated our result that the polygon group is better at identifying generators (order 6: $\beta$ 95\%-CI = $[0.35,1.72]$, $p < 0.05$; order 9: $\beta$ 95\%-CI = $[0.22,1.22]$, $p < 0.05$). We failed to replicate our result that the modular group is better than the polygon group at T/F questions about generators, in fact the polygon group performed significantly better in this experiment ($\beta$ 95\%-CI = $[0.004,0.59]$, $p < 0.05$). We failed to replicated our result that the hybrid group is significantly better than the modular group at identifying generators in either group order, although both effects were slightly in that direction (order 6: $\beta$ 95\%-CI = $[-1.11,1.27]$, $p > 0.05$; order 9: $\beta$ 95\%-CI = $[-0.78, 1.17]$, $p > 0.05$). (See Figure \ref{ex3_gen}.)
\begin{figure}[H]
\centering
\begin{subfigure}[c]{0.4\textwidth}
\centering
\includegraphics[width=\textwidth]{figures/3/gen_T_r.png}
\end{subfigure}
~
\begin{subfigure}[c]{0.4\textwidth}
\centering
\includegraphics[width=\textwidth]{figures/3/gen_TF_r.png}
\end{subfigure}
\caption{Experiment 3 -- Generator Results}
\label{ex3_gen}
\end{figure}\noindent 
\subsubsection{Which-Representation Question Results}
%%TODO: this
\subsection{Discussion}
With the exception of the T/F questions, we replicated all our results about the differences between the polygon and modular representations in at least one group order. Thus, this data continues to support the general hypothesis that the representations affect understanding of concepts built upon them. However, the data from this experiment do not seem to support the hypothesis that the hybrid group was achieving maximal performance. Instead, it appears more likely that the hybrid group was averaging, which could suggest a  number of possibilities, such as some subjects being confused or overloaded and performing worse, or subjects just picking a representation and using it exclusively. 
%%TODO: elaborate
\section{Meta-analysis}
To elucidate the pattern of results we observed, we repeated the bootstrap analysis we performed for experiment 3 on the data from experiments 1 \& 2, and performed a meta-analysis to produce better estimates of the effect sizes for our main effects of interest, namely the performance on inverses of zero and non-zero elements, on identifying generators, and on the T/F questions about generators.\\[11pt]
We estimated the positive effect of the polygon condition on inverse of zero questions to be large for both group orders, although it shows a decline consistent with some learning in the modular group (order 6: log OR = 3.01; order 9: log OR = 2.56; see fig. \ref{meta_inZ_p}). We estimated the negative effect of the polygon condition on inverse of non-zero questions to be moderately sized (order 6: log OR = -0.53; order 9: log OR = -0.81; see fig. \ref{meta_inNZ_p}). We estimated the positive effect of the polygon condition on identifying generators to be moderately sized (order 6: log OR = 0.68; order 9: log OR = 0.80; see fig. \ref{meta_genT_p}). We estimated that any effect of the polygon condition on answering True/False questions about generators is small (log OR = -0.14; see fig. \ref{meta_genTF_p}). Overall, these results support our claim that the polygon and modular representations produce differential understanding of different aspects of the groups.
\begin{figure}[H]
\centering
\begin{subfigure}[c]{0.4\textwidth}
\centering
\includegraphics[width=\textwidth]{figures/meta/p_in_Z_6.png}
\caption{Order 6}
\end{subfigure}
~
\begin{subfigure}[c]{0.4\textwidth}
\centering
\includegraphics[width=\textwidth]{figures/meta/p_in_Z_9.png}
\caption{Order 9}
\end{subfigure}
\caption{Meta Analysis -- Polygon vs. Modular, Inverse of Zero}
\label{meta_inZ_p}
\end{figure}\noindent 

\begin{figure}[H]
\centering
\begin{subfigure}[c]{0.4\textwidth}
\centering
\includegraphics[width=\textwidth]{figures/meta/p_in_NZ_6.png}
\caption{Order 6}
\end{subfigure}
~
\begin{subfigure}[c]{0.4\textwidth}
\centering
\includegraphics[width=\textwidth]{figures/meta/p_in_NZ_9.png}
\caption{Order 9}
\end{subfigure}
\caption{Meta Analysis -- Polygon vs. Modular, Inverse of Non-zero}
\label{meta_inNZ_p}
\end{figure}\noindent 

\begin{figure}[H]
\centering
\begin{subfigure}[c]{0.4\textwidth}
\centering
\includegraphics[width=\textwidth]{figures/meta/p_gen_T_6.png}
\caption{Order 6}
\end{subfigure}
~
\begin{subfigure}[c]{0.4\textwidth}
\centering
\includegraphics[width=\textwidth]{figures/meta/p_gen_T_9.png}
\caption{Order 9}
\end{subfigure}
\caption{Meta Analysis -- Polygon vs. Modular, Identifying Generators}
\label{meta_genT_p}
\end{figure}\noindent 

\begin{figure}[H]
\centering
\begin{subfigure}[c]{0.4\textwidth}
\centering
\includegraphics[width=\textwidth]{figures/meta/p_gen_TF_n.png}
\end{subfigure}
\caption{Meta Analysis -- Polygon vs. Modular, Generator T/F}
\label{meta_genTF_p}
\end{figure}\noindent 
We estimated the positive effect of the hybrid condition on inverse of zero questions to be large for both group orders, although it is not as large as that of the polygon condition, and shows a decline consistent with some learning in the modular group (order 6: log OR = 1.80; order 9: log OR = 1.40; see fig. \ref{meta_inZ_h}). We estimated the negative effect of the hybrid condition on inverse of non-zero questions to be moderately sized and possibly vanishing after further practice in the order 9 group (order 6: log OR = -0.58; order 9: log OR = -0.16; see fig. \ref{meta_inNZ_h}). We estimated the positive effect of the hybrid condition on identifying generators to be small in the order 6 group, but increasing in the order 9 group, (order 6: log OR = 0.19; order 9: log OR = 0.65; see fig. \ref{meta_genT_h}). We estimated that any effect of the hybrid condition on answering True/False questions about generators is small (log OR = -0.05; see fig. \ref{meta_genTF_h}). Taken together, these results suggest that although the hybrid group is not immediately achieving greater understanding across all aspects and concepts, after moving to the order 9 group it may be. However, it is still not achieving truly best-of-both-worlds performance, so there is room for further improvement. 
%%TODO: further discussion?
\begin{figure}[H]
\centering
\begin{subfigure}[c]{0.4\textwidth}
\centering
\includegraphics[width=\textwidth]{figures/meta/h_in_Z_6.png}
\caption{Order 6}
\end{subfigure}
~
\begin{subfigure}[c]{0.4\textwidth}
\centering
\includegraphics[width=\textwidth]{figures/meta/h_in_Z_9.png}
\caption{Order 9}
\end{subfigure}
\caption{Meta Analysis -- Hybrid vs. Modular, Inverse of Zero}
\label{meta_inZ_h}
\end{figure}\noindent 

\begin{figure}[H]
\centering
\begin{subfigure}[c]{0.4\textwidth}
\centering
\includegraphics[width=\textwidth]{figures/meta/h_in_NZ_6.png}
\caption{Order 6}
\end{subfigure}
~
\begin{subfigure}[c]{0.4\textwidth}
\centering
\includegraphics[width=\textwidth]{figures/meta/h_in_NZ_9.png}
\caption{Order 9}
\end{subfigure}
\caption{Meta Analysis -- Hybrid vs. Modular, Inverse of Non-zero}
\label{meta_inNZ_h}
\end{figure}\noindent 

\begin{figure}[H]
\centering
\begin{subfigure}[c]{0.4\textwidth}
\centering
\includegraphics[width=\textwidth]{figures/meta/h_gen_T_6.png}
\caption{Order 6}
\end{subfigure}
~
\begin{subfigure}[c]{0.4\textwidth}
\centering
\includegraphics[width=\textwidth]{figures/meta/h_gen_T_9.png}
\caption{Order 9}
\end{subfigure}
\caption{Meta Analysis -- Hybrid vs. Modular, Identifying Generators}
\label{meta_genT_h}
\end{figure}\noindent 

\begin{figure}[H]
\centering
\begin{subfigure}[c]{0.4\textwidth}
\centering
\includegraphics[width=\textwidth]{figures/meta/h_gen_TF_n.png}
\end{subfigure}
\caption{Meta Analysis -- Hybrid vs. Modular, Generator T/F}
\label{meta_genTF_h}
\end{figure}\noindent 
\section{Conclusion}
We explored the way representation of concepts in math instruction affects understanding of the concept being exemplified, and of more abstract concepts built upon it, using elementary group theory as our test domain. We found that even if two representations produce equal performance on a basic concept, they can produce differential understanding of the more abstract concepts related to it. Furthermore, it does not appear that there is always a clear advantage of one representation over another, instead a representation may be more useful for some types of questions, and less useful for others. We think that these findings provide a contribution to the discussion of representation in math cognition, illustrating that representations have different advantages and disadvantages. This suggests that the pursuit of a single best type of representation may be futile. \\[11pt]
Instead, we have identified an alternative strategy for improving performance: namely teaching multiple representations while encouraging subjects to integrate them. Preliminary data suggests that this may have positive effects even if the total instruction time is approximately the same. We think this provides an exciting new direction for research in both math cognition and math pedagogy. 
 
\bibliography{fyp_writeup_citations}

\bibliographystyle{apalike}

\newpage
\setcounter{secnumdepth}{-1}
\section{Appendix A: A Brief, Selective Introduction to Group Theory}
Groups are mathematical structures that provide us with a nice way of doing something like arithmetic with things besides the ordinary numbers, like symmetries of an object or permutations, or with smaller sets of ordinary numbers (as in this paper). They have applications throughout mathematics, physics, chemistry, and computer science. Here I present the formal definition of a group with informal intuitions in italics. A \textbf{group} consists of a set $G$ (\emph{some objects}) and a binary operation $*: G\times G \rightarrow G$ (\emph{a way of combining two objects to get another object}) such that:  
\begin{itemize}
\item $G$ is \textbf{closed} under $*$, that is $a*b \in G$ for all $a,b \in G$. (\emph{Combining two of the objects you started with gives you another of the objects you started with.}) 
\item $*$ is \textbf{associative}, $a*(b*c) = (a*b)*c$ for all $a,b,c \in G$. (\emph{It doesn't matter how you parenthesize the operation, just like addition or multiplication.})
\item There is an \textbf{identity} element $e \in G$ such that $\forall x \in G, e*x = x*e = x$. (\emph{There's something that when you combine it with anything else has no effect, just like multiplying by one gives you the same number back.})
\item Each element $x \in G$ has an \textbf{inverse} element $x^{-1} \in G$ such that $x*x^{-1} = x^{-1}*x = e$. (\emph{There's something you can combine with each element to get back to the identity, just like $2 \times 0.5 = 1$.})
\end{itemize}
For example, if we take $G$ to be the numbers less than $4$, $G = \{0,1,2,3\}$, and define a new operation $*$ by $$a*b = \begin{cases} a+b & \text{if } a+b < 4 \\ a+b-4 & \text{if } a+b \geq 4 \end{cases}$$
$G$ and $*$ form a group, called the \textbf{cyclic group of order 4}. For example, in this group $1*1 = 2$, $2 * 3 = 5-4 = 1$ because $5 \geq 4$, $3*1 = 4-4 = 0$, etc. $0$ is the identity in this group, because $0*x = x*0 = x$ for any of $0,1,2,3$. Furthermore, the inverse of $1$ in the group is $3$, because $1*3 = 4-4 = 0$, the inverse of $2$ is $2$, and so on.\\[11pt]
There is a great deal of structure to groups, far more than there is space to explain here. The only topic of interest for us beyond these simple properties will be the concept of \textbf{generators}. An element $x$ generates a group if every other element of the group can be written as $x*x*\cdots*x$ for some number of $x$s. For example, in our cyclic group of order 4, defined above, 1 is a generator of the group because $1 = 1, 2 = 1 * 1, 3 = 1 * 1 * 1, 0 = 1 * 1 * 1 * 1$. Similarly, 3 is a generator because $3 = 3, 2 = 3 * 3, 1 = 3 * 3 * 3, 0 = 3 * 3 * 3 * 3$. However, 2 is not a generator because $2 = 2, 0 = 2 * 2$, but there is no way to generate 1 or 3 using 2. This illustrates the only theorem we will give here: \\[11pt]
\textbf{Cyclic Group Generators Theorem:} In a cyclic group of order $n$, written as the integers $0$ to $n-1$, $x < n$ generates the group if and only if $x$ and $n$ are relatively prime (i.e. have no common factors except 1). \\[11pt] 
For more information on groups and group theory, see e.g. \cite{Lang2002}.
 
\end{document} 
